\documentclass[notheorems, aspectratio=169, 12pt, unicode]{beamer}
\usepackage{mysettings}
\begin{document}

\begin{frame}
 \titlepage
\end{frame}

\section{Introduction}

\begin{frame}<handout:0>{自己紹介}
 
\end{frame}

\section{Separation Logic}

\begin{frame}{分離論理とは?}
 
\end{frame}

\begin{frame}{ポインタとは?}
 
\end{frame}

\begin{frame}{分離論理の論理式}
 
\end{frame}

\begin{frame}{分離論理の意味のイメージ}{$\mapsto$}
 
\end{frame}

\begin{frame}{分離論理の意味のイメージ}{$\mathsepstar$}
 
\end{frame}

\begin{frame}{分離論理の意味のイメージ}{$\mathprls{\mathdash}{\mathdash}$}
 
\end{frame}

\begin{frame}{分離論理の意味のイメージ}{$\mathprtree{\mathdash}$}
 
\end{frame}

\begin{frame}{分離論理の意味のイメージ}{$\mathmagicwand$}
 
\end{frame}

\section{Program verification}

\begin{frame}{プログラム検証とは?}
 
\end{frame}

\begin{frame}{プログラム検証の例}
 
\end{frame}

\section{References}

\begin{frame}{参考文献}
 \begin{thebibliography}{9}
  \bibitem{Reynolds2002} John C. Reynolds, ``Separation Logic: A Logic for Shared Mutable Data Structure'', Proceedings of the 17th Anual IEEE Symposium on Logic in Computer Science, 2002.
  \bibitem{Brotherston2015} James Brotherston,
	  `` An introduction to separation logic '', 
	  Logic Summer School, ANU, 7 December 2015.
\bibitem{Sokratesnil2020} Sokratesnil、『分離論理入門のようなもの』、\url{}
 \end{thebibliography} 
\end{frame}

\appendix

\begin{frame}{Frame Rule}
 
\end{frame}

\end{document}
